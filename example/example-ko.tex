\documentclass[ko]{snu-ling-ba-thesis}


%%%%%%%%%%%%%%%%%%%%%%%%%%%%%%%%%%%%%%%%%%%%%%%%%%%%%%%%%%%%%%%%%%%%%%%%%%%%%%%
%%%%%%%%%%%%%%%%%%%%%%%%%%%%%%%%% Font setup %%%%%%%%%%%%%%%%%%%%%%%%%%%%%%%%%%
%%%%%%%%%%%%%%%%%%%%%%%%%%%%%%%%%%%%%%%%%%%%%%%%%%%%%%%%%%%%%%%%%%%%%%%%%%%%%%%


\usepackage{microtype}

% Works for pdfLaTeX, XeLaTeX, and LuaLaTeX
\usepackage{libertine}
\makeatletter
\setmainfont{\libertine@base}[
  Extension = .otf,
  Numbers = {\libertine@figurealign,\libertine@figurestyle},
  UprightFont = *_R,
  ItalicFont = *_RI,
  BoldFont = *_\libertine@boldstyle,
  BoldItalicFont = *_\libertine@boldstyle I,
  SlantedFont = *_R,
  SlantedFeatures = {FakeSlant=0.17},
]
\makeatother

% Math/Mono font setup
\usepackage[
  math-style=ISO,
  warnings-off={mathtools-colon, mathtools-overbracket},
]{unicode-math}
\setmathfont{LibertinusMath-Regular.otf}[Scale = MatchUppercase]

\setmonofont{JetBrainsMonoNerdFont}[
  Extension = .ttf,
  UprightFont = *-Regular,
  BoldFont = *-Bold,
  ItalicFont = *-Italic,
  BoldItalicFont = *-BoldItalic,
  Scale=MatchLowercase,
]

% Hangul font setup
\setmainhangulfont{KoPubWorldBatang_Pro}[
  Scale = MatchUppercase,
  UprightFont={* Light},
  BoldFont={* Bold},
  AutoFakeSlant = 0.15
]
\setsanshangulfont{KoPubWorldDotum_Pro}[
  Scale = MatchUppercase,
  BoldFont={* Bold},
]


%%%%%%%%%%%%%%%%%%%%%%%%%%%%%%%%%%%%%%%%%%%%%%%%%%%%%%%%%%%%%%%%%%%%%%%%%%%%%%%
%%%%%%%%%%%%%%%%%%%%%%%%%%%%%%%%%% Packages %%%%%%%%%%%%%%%%%%%%%%%%%%%%%%%%%%%
%%%%%%%%%%%%%%%%%%%%%%%%%%%%%%%%%%%%%%%%%%%%%%%%%%%%%%%%%%%%%%%%%%%%%%%%%%%%%%%


% Add your packages here, e.g.,
% \usepackage{tikz}
\usepackage{siunitx}

% For lorem ipsum; remove this line when writing your thesis
\usepackage{jiwonlipsum}

% hyperref *must* be the last package to be loaded!
\usepackage[pdfusetitle]{hyperref}


%%%%%%%%%%%%%%%%%%%%%%%%%%%%%%%%%%%%%%%%%%%%%%%%%%%%%%%%%%%%%%%%%%%%%%%%%%%%%%%
%%%%%%%%%%%%%%%%%%%%%%%%%%%%%%%%%% Metadata %%%%%%%%%%%%%%%%%%%%%%%%%%%%%%%%%%%
%%%%%%%%%%%%%%%%%%%%%%%%%%%%%%%%%%%%%%%%%%%%%%%%%%%%%%%%%%%%%%%%%%%%%%%%%%%%%%%


\addbibresource{bib.bib}

\title{서울대학교 언어학과 학사학위논문 템플릿}

% If there is no subtitle, just leave the argument empty, e.g.,
% \subtitle{}
\subtitle{국문 예시}

\author{박준영}
\affil{서울대학교 인문대학 언어학과}
\advisor{다크시푸트라 파니니}
\date{2025년 8월}


\begin{document}
\maketitle

\pagenumbering{roman}
\begin{abstract}
  \jiwon[1]
\end{abstract}


\tableofcontents


\section{서론}\label{sec:introduction}
\pagenumbering{arabic}

본 템플릿의 구성은 다음과 같다.
\ref{sec:body}장 본론의 \ref{subsec:picture}절에서 그림의 예시를 보여준다.
\ref{subsec:table}절에서 표의 예시를 보여준다.
\ref{sec:conclusion}장에서는 본 템플릿을 요약한다.
\ref{sec:references}장과 \ref{sec:appendices}장에서는 각각 참고 문헌과 부록의 예시를 보여준다.

\subsection{절 예시}\label{subsec:section}
\jiwon[2-3]


\section{본론}\label{sec:body}
정보 엔트로피는 각 메시지에 포함된 정보의 기댓값으로 식~\eqref{eq:entropy}\와 같다(\cite{sha48}).
\begin{equation}\label{eq:entropy}
  H(X) = -\sum_{i=1}^n {\mathrm{P}(x_i) \log_b \mathrm{P}(x_i)}
\end{equation}

\jiwon[4-6]


\subsection{그림}\label{subsec:picture}
그림 예시는 그림~\ref{fig:example}\와 같다.
그림~\ref{fig:snu}\은 서울대학교 로고이고 그림~\ref{fig:ch}\는 서울대학교 인문대학 로고이다.

\begin{figure}[htp]
  \centering
  \begin{subfigure}[b]{0.5\textwidth}
    \centering
    \includegraphics[width=0.5\textwidth]{logo1.pdf}
    \caption{서울대학교 로고}\label{fig:snu}
  \end{subfigure}%
  \begin{subfigure}[b]{0.5\textwidth}
    \centering
    \includegraphics[width=0.9\textwidth]{logo2.pdf}
    \caption{서울대학교 인문대학 로고}\label{fig:ch}
  \end{subfigure}
  \caption{그림 예시}\label{fig:example}
\end{figure}

\jiwon[7-8]


\subsection{표}\label{subsec:table}
표 예시는 표~\ref{tab:example}\과 같다.\footnote{\jiwon[12]}

\begin{table}[htp]
  \centering
  \caption{표 예시}\label{tab:example}
  \begin{tblr}{cc}
    \toprule
    상수 & 값 \\\midrule
    $c$ & \SI{299792458}{\meter\per\second} \\
    $h$ & \SI{6.62607015e-34}{\joule\per\hertz} \\\bottomrule
  \end{tblr}
\end{table}

\jiwon[9-10]


\section{결론}\label{sec:conclusion}
\jiwon[11]


\section{참고 문헌}\label{sec:references}
\printbibliography


\section{부록}\label{sec:appendices}
\appendix

\apxitem{첫 번째 부록}
\jiwon[12]

\end{document}
